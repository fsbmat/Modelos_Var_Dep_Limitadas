\documentclass[ignorenonframetext,]{beamer}
\usetheme{Warsaw}
\usepackage{amssymb,amsmath}
\usepackage{ifxetex,ifluatex}
\usepackage{fixltx2e} % provides \textsubscript
\ifxetex
  \usepackage{fontspec,xltxtra,xunicode}
  \defaultfontfeatures{Mapping=tex-text,Scale=MatchLowercase}
\else
  \ifluatex
    \usepackage{fontspec}
    \defaultfontfeatures{Mapping=tex-text,Scale=MatchLowercase}
  \else
    \usepackage[utf8]{inputenc}
  \fi
\fi
\usepackage{color}
\usepackage{fancyvrb}
\DefineShortVerb[commandchars=\\\{\}]{\|}
\DefineVerbatimEnvironment{Highlighting}{Verbatim}{commandchars=\\\{\}}
% Add ',fontsize=\small' for more characters per line
\newenvironment{Shaded}{}{}
\newcommand{\KeywordTok}[1]{\textcolor[rgb]{0.00,0.44,0.13}{\textbf{{#1}}}}
\newcommand{\DataTypeTok}[1]{\textcolor[rgb]{0.56,0.13,0.00}{{#1}}}
\newcommand{\DecValTok}[1]{\textcolor[rgb]{0.25,0.63,0.44}{{#1}}}
\newcommand{\BaseNTok}[1]{\textcolor[rgb]{0.25,0.63,0.44}{{#1}}}
\newcommand{\FloatTok}[1]{\textcolor[rgb]{0.25,0.63,0.44}{{#1}}}
\newcommand{\CharTok}[1]{\textcolor[rgb]{0.25,0.44,0.63}{{#1}}}
\newcommand{\StringTok}[1]{\textcolor[rgb]{0.25,0.44,0.63}{{#1}}}
\newcommand{\CommentTok}[1]{\textcolor[rgb]{0.38,0.63,0.69}{\textit{{#1}}}}
\newcommand{\OtherTok}[1]{\textcolor[rgb]{0.00,0.44,0.13}{{#1}}}
\newcommand{\AlertTok}[1]{\textcolor[rgb]{1.00,0.00,0.00}{\textbf{{#1}}}}
\newcommand{\FunctionTok}[1]{\textcolor[rgb]{0.02,0.16,0.49}{{#1}}}
\newcommand{\RegionMarkerTok}[1]{{#1}}
\newcommand{\ErrorTok}[1]{\textcolor[rgb]{1.00,0.00,0.00}{\textbf{{#1}}}}
\newcommand{\NormalTok}[1]{{#1}}
% Comment these out if you don't want a slide with just the
% part/section/subsection/subsubsection title:
\AtBeginPart{\frame{\partpage}}
\AtBeginSection{\frame{\sectionpage}}
\AtBeginSubsection{\frame{\subsectionpage}}
\AtBeginSubsubsection{\frame{\subsubsectionpage}}
\setlength{\parindent}{0pt}
\setlength{\parskip}{6pt plus 2pt minus 1pt}
\setlength{\emergencystretch}{3em}  % prevent overfull lines
\setcounter{secnumdepth}{0}

\title{Getting Started with R}
\author{Randall Pruim}
\date{2012/05/06}

\begin{document}
\frame{\titlepage}

\begin{frame}\frametitle{Getting Started with R}

\begin{itemize}[<+->]
\item
  basic arithmetic
\item
  assignment and reuse
\item
  some short cuts
\item
  functions make R's world go 'round
\item
  data: -- vectors and data frames
\item
  graphics
\item
  formulas
\end{itemize}

\end{frame}

\begin{frame}[fragile]\frametitle{Basic Arithmetic \& Assignment}

\begin{Shaded}
\begin{Highlighting}[]
\DecValTok{3} \NormalTok{+ }\DecValTok{5}
\end{Highlighting}
\end{Shaded}

\begin{verbatim}
## [1] 8
\end{verbatim}

\begin{Shaded}
\begin{Highlighting}[]
\KeywordTok{log}\NormalTok{(}\DecValTok{10}\NormalTok{)}
\end{Highlighting}
\end{Shaded}

\begin{verbatim}
## [1] 2.303
\end{verbatim}

\begin{Shaded}
\begin{Highlighting}[]
\KeywordTok{exp}\NormalTok{(}\DecValTok{1}\NormalTok{)}
\end{Highlighting}
\end{Shaded}

\begin{verbatim}
## [1] 2.718
\end{verbatim}

\begin{Shaded}
\begin{Highlighting}[]
\KeywordTok{sin}\NormalTok{(pi/}\DecValTok{3}\NormalTok{)}
\end{Highlighting}
\end{Shaded}

\begin{verbatim}
## [1] 0.866
\end{verbatim}

\begin{Shaded}
\begin{Highlighting}[]
\NormalTok{x <- }\DecValTok{5}
\NormalTok{x^}\DecValTok{2}
\end{Highlighting}
\end{Shaded}

\begin{verbatim}
## [1] 25
\end{verbatim}

\end{frame}

\begin{frame}[fragile]\frametitle{Functions make R's world go 'round}

Syntax is easy: * function name * round parens * comma separated list of
arguments * arguments may be named or used positionally

\begin{Shaded}
\begin{Highlighting}[]
\KeywordTok{log}\NormalTok{(}\DecValTok{8}\NormalTok{, }\DataTypeTok{base =} \DecValTok{2}\NormalTok{)}
\end{Highlighting}
\end{Shaded}

\begin{verbatim}
## [1] 3
\end{verbatim}

\begin{Shaded}
\begin{Highlighting}[]
\KeywordTok{log}\NormalTok{(}\DecValTok{8}\NormalTok{)}
\end{Highlighting}
\end{Shaded}

\begin{verbatim}
## [1] 2.079
\end{verbatim}

\begin{Shaded}
\begin{Highlighting}[]
\KeywordTok{log}\NormalTok{(}\DecValTok{8}\NormalTok{, }\DataTypeTok{base =} \KeywordTok{exp}\NormalTok{(}\DecValTok{1}\NormalTok{))}
\end{Highlighting}
\end{Shaded}

\begin{verbatim}
## [1] 2.079
\end{verbatim}

\end{frame}

\begin{frame}\frametitle{Some Shortcuts}

\begin{itemize}[<+->]
\item
  Use arrow keys to recall previous commands
\item
  Use to quit current command without executing it
\item
  Click on stop sign to end long running process
\item
  History tab contains list of recent commands
\item
  Plots tab contains recent plots and has buttons to navigate among them
\item
  Get in the habit of putting commands into files -- save your work!
\end{itemize}

\end{frame}

\begin{frame}\frametitle{Data}

Most (perhaps all) of the data we will use in this class can be thought
of in a rectangular layout: - rows = observational units (cases,
subjects, observations, individuals, etc.) - columns = variables

Barring missing data, each observational unit will have a \textbf{value}
for each variable

\end{frame}

\begin{frame}[fragile]\frametitle{Data in R}

Typically store data in data framesHow

\begin{itemize}[<+->]
\item
  source editor: \href{http://www.rstudio.org/}{RStudio} (perfect
  integration with \href{http://yihui.name/knitr/}{\textbf{knitr}};
  one-click compilation); currently you have to use the version
  \textgreater{}= 0.96.109
\item
  HTML5 slides converter:
  \href{http://johnmacfarlane.net/pandoc/}{pandoc}; this document was
  generated by:
  \texttt{pandoc -s -S -i -t dzslides -{}-mathjax knitr-slides.md -o knitr-slides.html}
\item
  the file
  \href{https://github.com/yihui/knitr-examples/blob/master/009-slides.md}{\texttt{knitr-slides.md}}
  is the markdown output from its
  \href{https://github.com/yihui/knitr-examples/blob/master/009-slides.Rmd}{source}:
  \texttt{library(knitr); knit('knitr-slides.Rmd')}
\item
  or simple click the button \texttt{Knit HTML} in RStudio
\end{itemize}

\begin{Shaded}
\begin{Highlighting}[]
\KeywordTok{print}\NormalTok{(}\KeywordTok{sessionInfo}\NormalTok{(), }\DataTypeTok{locale =} \OtherTok{FALSE}\NormalTok{)}
\end{Highlighting}
\end{Shaded}

\begin{verbatim}
## R version 2.15.1 (2012-06-22)
## Platform: x86_64-apple-darwin9.8.0/x86_64 (64-bit)
## 
## attached base packages:
## [1] stats     graphics  grDevices utils     datasets  methods   base     
## 
## other attached packages:
## [1] ggplot2_0.9.1 knitr_0.6.3  
## 
## loaded via a namespace (and not attached):
##  [1] MASS_7.3-20        RColorBrewer_1.0-5 RCurl_1.91-1      
##  [4] Rcpp_0.9.12        XML_3.9-4          colorspace_1.1-1  
##  [7] dichromat_1.2-4    digest_0.5.2       evaluate_0.4.2    
## [10] formatR_0.5.1      grid_2.15.1        labeling_0.1      
## [13] memoise_0.1        munsell_0.3        parser_0.0-16     
## [16] plyr_1.7.1         proto_0.3-9.2      reshape2_1.2.1    
## [19] scales_0.2.1       stringr_0.6        tools_2.15.1      
\end{verbatim}

\end{frame}

\end{document}
